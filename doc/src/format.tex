\section{Image formats}
\label{sec:format}

VIPS has a simple system for adding support for new image file formats. 
You can ask VIPS to find a format to load a file with
or to select a image file writer based on a filename. Convenience functions
copy a file to an \verb+IMAGE+, or an \verb+IMAGE+ to a file. New formats may
be added to VIPS by simply defining a new subclass of \verb+VipsFormat+.

This is a parallel API to \verb+im_open()+, see \pref{sec:open}. The
format system is useful for images which are large or slow to open,
because you pass a descriptor to write to and so control how and where
the decompressed image is held. \verb+im_open()+ is useful for images in
formats which can be directly read from disc, since you will avoid a copy
operation and can directly control the disc file. The inplace operations
(see \pref{sec:inplace}), for example, will only work directly on disc
images if you use \verb+im_open()+.

\subsection{How a format is represented}

See the man page for \verb+VipsFormat+ for full details, but briefly, an image
format consists of the following items:

\begin{itemize}
\item
A name, a name that can be shows to the user, and a list of possible filename
suffixes (\verb+.tif+, for example)

\item
A function which tests for a file being in that format, a function which loads 
just the header of the file (that is, it reads properties like width and
height and does not read any pixel data) and a function which loads the pixel
data

\item
A function which will write an \verb+IMAGE+ to a file in the format

\item
And finally a function which examines a file in the format and returns flags
indicating how VIPS should deal with the file. The only flag in the current
version is one indicating that the file can be opened lazily

\end{itemize}

A switch to the \verb+vips+ command-line program is handy for listing the
supported formats. Try:

\begin{verbatim}
vips --list classes
\end{verbatim}

\noindent
And look for subclasses of \verb+VipsFormat+.

\subsection{The format class}

The interface to the format system is defined by the abstract base class
\verb+VipsFormat+. Formats subclass this and implement some or all of the
methods. Any of the functions may be left NULL and VIPS will try to make do 
with what you do supply. Of course a format with all functions as NULL will 
not be very useful.

As an example, \fref{fg:newformat} shows how to register a new format in a
plugin.

\begin{fig2}
\begin{verbatim}
static const char *my_suffs[] = { ".me", NULL };

static int
is_myformat( const char *filename )
{
  unsigned char buf[2];

  if( im__get_bytes( filename, buf, 2 ) &&
    (int) buf[0] == 0xff && 
    (int) buf[1] == 0xd8 )
    return( 1 );

  return( 0 );
}

// This format adds no new members.
typedef VipsFormat VipsFormatMyformat;
typedef VipsFormatClass VipsFormatMyformatClass;

static void
vips_format_myformat_class_init( VipsFormatMyformatClass *class )
{
  VipsObjectClass *object_class = (VipsObjectClass *) class;
  VipsFormatClass *format_class = (VipsFormatClass *) class;

  object_class->nickname = "myformat";
  object_class->description = _( "My format" );

  format_class->is_a = is_myformat;
  format_class->header = my_header;
  format_class->load = my_read;
  format_class->save = my_write;
  format_class->get_flags = my_get_flags;
  format_class->priority = 100;
  format_class->suffs = my_suffs;
}

static void
vips_format_myformat_init( VipsFormatMyformat *object )
{
}

G_DEFINE_TYPE( VipsFormatMyformat, vips_format_myformat, VIPS_TYPE_FORMAT );

char *
g_module_check_init( GModule *self )
{
  // register the type
  vips_format_myformat_get_type(); 
}
\end{verbatim}
\caption{Registering a format in a plugin}
\label{fg:newformat}
\end{fig2}

\subsection{Finding a format}

You can loop over the subclasses of \verb+VipsFormat+ in order of priority
with \verb+vips_format_map()+. Like all the map functions in VIPS, this take
a function and applies it to every element in the table until the function
returns non-zero or until the table ends.

You find an \verb+VipsFormatClass+ to use to open a file with
\verb+vips_format_for_file()+. This finds the first format whose \verb+is_a()+
function returns true or whose suffix list matches the suffix of the filename,
setting an error message and returning NULL if no format is found.

You find a format to write a file with \verb+vips_format_for_name()+. This
returns the first format with a save function whose suffix list matches the
suffix of the supplied filename.

\subsection{Convenience functions}

A pair of convenience functions, \verb+vips_format_write()+ and
\verb+vips_format_read()+, will copy an image to and from disc using the
appropriate format.
